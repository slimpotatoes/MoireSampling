% % % % % % %
% MoireSampling Problem Statement - Alexandre Pofelski
% Dec 2020
% % % % % % %

% Document class
\documentclass[
11pt,
a4paper,
oneside,
onecolumn,
final %Final visual
]{article}

% Packages
\usepackage[utf8]{inputenc} % Deal with international characters
\usepackage[T1]{fontenc} % Deal with international character 
\usepackage[binary-units=true]{siunitx} % Scientific units (micro-liter, etc)
\usepackage{graphicx} % Include images in text
\usepackage{enumerate} % Have access to bullet points
\usepackage{float} % Position the figures with H (floating point)
\usepackage{amsmath, mathtools, amsfonts, amssymb} % Equations and math symbol
\usepackage{bm} % Bold font in math expression
\usepackage{array} % Tables
\usepackage{longtable} % Tables through multiple pages
\usepackage{booktabs} % Visual parameter on tables
\usepackage{multirow}

% Referencing
\usepackage{hyperref} % Style for referencing
\usepackage{cleveref} % Internal referencing system + citing biblio
\usepackage{csquotes} % Quoting sentences

% Page style (think about memoir documentation if printing is not good)
\usepackage[top=2.5cm, bottom=2.5cm, left=3cm, right=3cm, includeheadfoot]{geometry}

% Text font, style and rules
\renewcommand{\arraystretch}{1.2} % Rule to apply on all table (stretch array)
\renewcommand{\familydefault}{ppl} % font generic text
\usepackage{xcolor} % Custom color
\definecolor{color_citation}{RGB}{50,170,50}
\hypersetup{
    %bookmarks=true,% show bookmarks bar?
    colorlinks=true, % false: boxed links; true: colored links
    linkcolor=red, % color of internal links (change box color with %linkbordercolor)
    citecolor=color_citation, % color of links to bibliography
    filecolor=magenta, % color of file links
    urlcolor=blue % color of external links
}
\usepackage{indentfirst} % Indent at the begining of the paragraph automatically
\setcounter{secnumdepth}{4} % numbering to subsubsections
\setcounter{tocdepth}{4} % table of content showing up to the level of subsubsection
\newcommand{\myparagraph}[1]{\paragraph{#1}\mbox{}\\} % Start a new line when doing a paragraph

\setlength{\textfloatsep}{10pt plus 1.0pt minus 2.0pt}
\setlength{\floatsep}{10pt plus 1.0pt minus 1.0pt}
\setlength{\intextsep}{10pt plus 1.0pt minus 1.0pt}
\newenvironment{tight_enumerate}{
\begin{enumerate}
  \setlength{\itemsep}{0pt}
  \setlength{\parskip}{0pt}
}{\end{enumerate}}

\begin{document}

\title{\huge Problem Statement : \\ MoireSampling}
\author{slimpotatoes}
\maketitle

\clearpage

\tableofcontents
\clearpage

\section{Introduction}

In digital processing context, sampling refers to the process of evaluating a function through successive local measurements. As a continuous function cannot be measured on the entire continuum, a finite set of locations must be chosen for the sampler. The choice of the sampling locations has a direct impact on the representation of the function and could miss represent it. For example, too spaced sampling locations might miss a local variation of the function. Based on sampling theory sets of rules can be defined for the sampler to properly represent certain types of functions. The famous Whittaker Nyquist Kotel'nikov Shannon (WNKS) sampling theorem sets a condition on a periodic sampler to properly represent a bandwidth limited function \cite{Whittaker1910,Nyquist1928,Kotelnikov1933,Shannon1949}. In that case, the sampler is said to be \emph{oversampling} the function. Else, the representation of the function is distorted and the sampler is said to be \emph{undersampling} the same function. The undersampling distortions is commonly referred as aliasing artefacts and is mostly unwanted. However, on periodic functions, the undersampling sampler generates a periodic representation carrying valuable information \cite{Pofelski2020}. The intentional use of undersampling samplers on periodic functions is referred as Moir{\'e} sampling as an analogy to the Moir{\'e} interference from two superimposed periodic patterns. The MoireSampling software comes to play in the context of using Moire sampled signals to determine properties of the original function evaluated.


\section{Objective}

The objective of the MoireSampling software is to simulate the undersampling sampling process of a sampler coherently interfering with a periodic lattice. Providing the samplers properties and the periodic structure, the MoireSampling software will provide 
the undersampling results and track the distortions of the periodic lattice. 

\section{Interests}

As the Moire sampled signal is a distortion of the original function sampled, the direct interpretation of the undersampled signal is a complex task. The MoireSampling software provides to the user the necessary information to interpret the undersampled signal and potentially recover the original function without loss of information. In addition, the MoireSampling software can be used to design a Moire sampling experiment by simulating the distortions process, since some distortions can be favourable in a certain context. Finally, a variety of samplers can be tested with MoireSampling to explore alternative methods to measure a periodic lattice. 

\bibliographystyle{IEEEtran}
\bibliography{../Ref/MoireSampling_biblio.bib}

\end{document}